%% LyX 2.1.2.1 created this file.  For more info, see http://www.lyx.org/.
%% Do not edit unless you really know what you are doing.
\documentclass[oneside]{amsart}
\usepackage[latin9]{inputenc}
\usepackage{geometry}
\geometry{verbose,lmargin=2.5cm,rmargin=2.5cm}
\usepackage{float}
\usepackage{enumitem}
\usepackage{amsthm}
\usepackage{amssymb}
\usepackage{graphicx}
\usepackage{esint}
\usepackage{cite}

\makeatletter

%%%%%%%%%%%%%%%%%%%%%%%%%%%%%% LyX specific LaTeX commands.
%% A simple dot to overcome graphicx limitations
\newcommand{\lyxdot}{.}


%%%%%%%%%%%%%%%%%%%%%%%%%%%%%% Textclass specific LaTeX commands.
\numberwithin{equation}{section}
\numberwithin{figure}{section}
\theoremstyle{plain}
\newtheorem{thm}{\protect\theoremname}
  \theoremstyle{plain}
  \newtheorem{conjecture}[thm]{\protect\conjecturename}
  \theoremstyle{plain}
  \newtheorem*{thm*}{\protect\theoremname}
  \theoremstyle{plain}
  \newtheorem{fact}[thm]{\protect\factname}
 \newlist{casenv}{enumerate}{4}
 \setlist[casenv]{leftmargin=*,align=left,widest={iiii}}
 \setlist[casenv,1]{label={{\itshape\ \casename} \arabic*.},ref=\arabic*}
 \setlist[casenv,2]{label={{\itshape\ \casename} \roman*.},ref=\roman*}
 \setlist[casenv,3]{label={{\itshape\ \casename\ \alph*.}},ref=\alph*}
 \setlist[casenv,4]{label={{\itshape\ \casename} \arabic*.},ref=\arabic*}
  \theoremstyle{definition}
  \newtheorem{example}[thm]{\protect\examplename}

%%%%%%%%%%%%%%%%%%%%%%%%%%%%%% User specified LaTeX commands.
\usepackage{cite}




\newtheorem{theorem}{}\newtheorem{axiom}[theorem]{}\newtheorem{corollary}{}[theorem]

\newcommand{\jk}[2]{ J_{k-1}\left (4 \pi \frac{ \sqrt{ #1 } }{ #2 } \right) }
\newcommand{\er}{E}
\DeclareMathOperator*{\res}{Res}

\DeclareMathOperator{\li}{li}



  \providecommand{\claimname}{Claim}
  \providecommand{\conjecturename}{Conjecture}
  \providecommand{\remarkname}{Remark}
\providecommand{\theoremname}{Theorem}

\makeatother

  \providecommand{\conjecturename}{Conjecture}
  \providecommand{\examplename}{Example}
  \providecommand{\factname}{Fact}
  \providecommand{\theoremname}{Theorem}
 \providecommand{\casename}{Case}
\providecommand{\theoremname}{Theorem}

\begin{document}

\title{Adiabatic Processes}


\author{Helang Liu, Ruizhi Deng}


\date{March 1, 2015}

\maketitle

\section{Introduction}

The adiabatic process happens in physical phenomena including oscillation
and thermal transmssion without transfer of heat and matter between
a system and its surroundings. This project focuses on the adiabatic
process of an osciallating system. For a single oscillator, we have
the oscillation equation 
\[
\ddot{x}+\omega(t)^{2}x=0.
\]
The energy of the system can be described by the function 
\[
E(t)=\frac{(x'(t))^{2}}{2}+\frac{\omega^{2}(t)(x(t))^{2}}{2}.
\]
In the equation and function above, $\omega(t)$ is the angular velocity
and is directly proportional to frequency. $x(t)$ is the displacement,
and $E(t)$ is the energy of the whole system. This project explores
how $x$ and $E$ behave with different oscillating functions.


\section{special cases of $\omega$}

Let's first consider the simplest oscillating system, the pendulum.
In this speicial case, we assume the initial velocity is nonezero
and $\omega(t)$ is a constant. Suppose $\omega(t)=\omega_{0}$, then
$\omega(t)^{2}=\omega_{0}^{2}$, $x''(t)+\omega_{0}^{2}x=0$, we have

\[
x''(t)=-\omega(t)^{2}x(t)=-\omega_{0}^{2}x(t).
\]
Solve the equation and we get\begin{center} 
\begin{eqnarray*}
x(t) & = & c_{1}\sin(|\omega_{0}|t)+c_{2}\sin(|\omega_{0}|t)\\
x'(t) & = & c_{1}|\omega_{0}|\sin(|\omega_{0}|t)-c_{2}|\omega_{0}|\cos(|\omega_{0}|t)\\
x''(t) & = & -\omega_{0}^{2}(c_{1}\cos t+c_{2}\cos t)=-c_{1}\omega_{0}^{2}\cos t-c_{2}\omega_{0}^{2}\sin t
\end{eqnarray*}
\end{center} for some $c_{1},c_{2}\in R$. The energy of the system
$E=\frac{(x'(t))^{2}}{2}+\frac{\omega_{0}^{2}(x(t))^{2}}{2}$ and
$E'=x'(t)x''(t)+\omega_{0}^{2}x(t)x'(t)=0$. 

Hence, if $\omega$ is a constant, the displacement, $x$, is periodic
and the energy $E$ is also a constant. The period the displacement
function will decrease if $\omega$increases. This result also makes
sense in the physical context. If the frequency always remains the
same, there should be no unbalanced external forces that drive the
system and the total energy definitely won't change.

For other cases of $\omega(t)$, write a program in Python to simulate
the motion of the oscillator and we record $\omega^{2}$ and $E$
every 0.1 second.

If $\omega(t)=\epsilon\cdot t+\omega_{0}$ for some $\epsilon\neq0$,
the function $\omega(t)$ is linear. We find out that $E(t)$ is not
a linear function about t. We set $\omega_{0}=5$ and $\varepsilon=0.1$
Below are the scatter plots of Energy vs Time and Energy vs Square
of angular velocity:

\begin{figure}[H]
\begin{centering}
\protect\caption{$\omega(t)=5+0.1\times t$}

\par\end{centering}

\centering{}\includegraphics[width=6cm,height=5cm]{li_EvT}\includegraphics[width=6cm,height=5cm]{li_EvS}
\end{figure}


We can easily see that energy function of time t can approximate by
a quadratic function, $E(t)=0.06t^{2}+2.4t+24.016$ and the relation
between $E$ and $\omega^{2}$ can be approximated by $E=6.0017\omega^{2}+0.029$.

We repeat this process for by setting $\omega=\sqrt{5+0.1t},\omega=\sin t,\omega=\sin t+3,\,\mbox{and }\omega=e^{t}$.
Below are the scatter plots of and Energy vs time and Energy vs Square
of angular velocity based on the data we collected.

\setlength{\parindent}{0pt}

~

\begin{figure}[H]
\begin{centering}
\protect\caption{$\omega(t)=\sqrt{5+0.1\times t}$}

\par\end{centering}

\centering{}\includegraphics[width=6cm,height=5cm]{sq_EvT}\includegraphics[width=6cm,height=5cm]{sq_EvS}
\end{figure}


\begin{figure}[H]
\begin{centering}
\protect\caption{$\omega(t)=\sin(t)$}

\par\end{centering}

\centering{}\includegraphics[width=6cm,height=5cm]{si_EvT}\includegraphics[width=6cm,height=5cm]{si_EvS}
\end{figure}


\begin{figure}[H]
\begin{centering}
\protect\caption{$\omega(t)=\sin(t)+3$}

\par\end{centering}

\centering{}\includegraphics[width=6cm,height=5cm]{s3_EvT}\includegraphics[width=6cm,height=5cm]{s3_EvS}
\end{figure}


\begin{figure}[H]
\begin{centering}
\protect\caption{$\omega(t)=e^{t}$}

\par\end{centering}

\centering{}\includegraphics[width=6cm,height=5cm]{ex_EvT}\includegraphics[width=6cm,height=5cm]{ex-EvS}
\end{figure}



\section{Discoveries and Conjectures:}

\ \setlength{\parindent}{15pt}

As can be seen from most of the scatter plots of Energy vs Square
of Frequency above, the relation between $E$ and $\omega^{2}$ can
be approximated by a linear relation. Notice that there's no clear
relation between $E$ and $\omega^{2}$ in the scatter plots of Energy
vs Square of Frequency with $\omega=\sin(t)$. However, if we replace
$\omega(t)=\sin t$ with $\omega(t)=\sin t+3$, the relation between
$E$ and $\omega^{2}$ is almost linear again. We also feel it necessary
to point out that in the scatter plots of Energy vs Square of Frequency
with $\omega=e^{t}$ the relation between $E$ and $\omega^{2}$ deviates
from the linear relation as E gets larger but this deviation might
be caused by the inaccuracy of doing integration in our program.

Hence we put forward our our first conjecture. 
\begin{conjecture}
The relation between $E$ and $\omega^{2}$ is a linear relation,
$i.e.\, E(t)=a\cdot\omega^{2}(t)+b$.\end{conjecture}
\begin{thm}
$E(t)=a\cdot w^{2}(t)+b$ if and only if $x(t)$ is a constant function.\end{thm}
\begin{proof}
Recall that $E=\frac{(x')^{2}}{2}+\frac{\omega^{2}x^{2}}{2}$. Then
we have 
\begin{alignat*}{1}
E' & =x'x''+\omega\omega'x^{2}+\omega^{2}xx'\\
 & =x'(\omega^{2}x+x'')+\omega\omega'x^{2}\\
 & =0+\omega\omega'x^{2}\\
 & =\omega\omega'x^{2}.
\end{alignat*}
Since $E(t)=a\cdot w^{2}(t)+b$, $E'=2a\cdot\omega\omega'$. Hence
$2a=x^{2}$, $x=\sqrt{2a}$ is a constant. Therefore, if $E(t)=a\cdot w^{2}(t)+b$
then $x(t)$ a is constant function. Conversely, suppose $x(t)$ is
a constant function, then $x^{2}(t)$ is a constant function as well.
Recall that $E'=\omega\omega'x^{2}=(\omega^{2})'\cdot\frac{x^{2}}{2}$.
If $\frac{x^{2}}{2}$ is a constant, taking the integral on both sides
yields $E(t)=\frac{x^{2}}{2}\omega(t)^{2}+c$ where $c$ is a constant. 
\end{proof}
Note that it's impossible that $E\propto\omega^{2}$ if $x(t)$ is
a non-constant function. Hence our first conjecture can not describe
the relation between $E$ and $\omega^{2}$ precisely in any case
other than the trivial case of constant $\omega$. However in most
of the plots above with $\omega(t)$ non-constant, all the points
of $(\omega^{2},E)$ are scattered along some straighty line. Therefore
we contemplate that $E(\omega^{2})$ are bounded by some linear functions
of $\omega^{2}$. This could happen if $\omega\omega'\geqslant0,\forall t$
or $\omega\omega'\leqslant0,\forall t$ and $x$ was bounded or in
other words $x^{2}$ is bounded. If $a\leqslant x^{2}\leqslant b$
for some $a,b\geqslant0$ and $(\omega^{2})'\geqslant0$. We have
$a\cdot(\omega^{2})'\leqslant(\omega^{2})'x^{2}\leqslant b\cdot(\omega^{2})'$.
If $(\omega^{2})'=2\omega\omega'<0$, we have $b\cdot(\omega^{2})'\leqslant(\omega^{2})'x^{2}\leqslant a\cdot(\omega^{2})'$.
Taking the integral on each term yields $a\omega^{2}+c_{1}\leqslant E\leqslant b\omega^{2}+c_{2}$
if $(\omega^{2})'\geqslant0$ and $b\omega^{2}+c_{1}\leqslant E\leqslant a\omega^{2}+c_{2}$
if $(\omega^{2})'<0$ where $c_{1},c_{2}\in R$. 
\begin{thm}
If $k>0$ and $\omega(t)=e^{k\cdot t+b}$, then x is bounded.\end{thm}
\begin{proof}
Recall
\[
E'=\omega\omega'x^{2}.
\]
Then we have
\[
\begin{array}{ccc}
E' & = & e^{k\cdot t+b}\cdot k\cdot e^{k\cdot t+b}x^{2}\\
 & = & k\cdot\omega^{2}x^{2}
\end{array}.
\]
$k\cdot\omega^{2}x^{2}\leqslant2k\cdot(\frac{(x')^{2}}{2}+\frac{\omega^{2}x^{2}}{2})=2kE\Rightarrow E'\leqslant2kE$.
When $x\neq0$, we have $E>0$. Then we get $\frac{E'}{E}\leqslant2k$.
Integrate on both site and we have $\int_{0}^{t}\frac{E'}{E}dt\leqslant\int_{0}^{t}2k\, dt$
and $\log E(t)-\log E(0)\leqslant2kt$. $E(t)\leqslant E(0)\cdot e^{2kt}$.
Since $E(t)=\frac{(x')^{2}}{2}+\frac{\omega^{2}x^{2}}{2}\geqslant\frac{\omega^{2}x^{2}}{2}$,
we get $\frac{\omega^{2}x^{2}}{2}\leqslant E(0)\cdot e^{2kt}\Rightarrow\frac{e^{2k\cdot t+2b}\cdot x^{2}}{2}\leqslant E(0)\cdot e^{2k\cdot t}$.
Hence we have $x^{2}\leqslant\frac{2E(0)}{e^{2b}}$ which shows $x$
is bounded.\end{proof}
\begin{conjecture}
Define $f(t)={\displaystyle \inf_{x\geqslant t}}\omega^{2}(x)$. If
$\int_{0}^{t}\omega^{2}(t)dt\neq0\,\forall t>0$ and ${\displaystyle \lim_{t\rightarrow\infty}f(t)>0}$,
$x(t)$ is bounded.
\end{conjecture}
We will show this statement is not correct in next section.


\section{Notes on $\omega(t)=\sin(kt)+c$}

For certain $k$ and $c$, $\sin(kt)+c$ might not suffice the condition
of the conjecture since they can reach 0 periodically, so the displacement
may or may not be bounded. We first consider $\omega(t)=\sin(t)$.
From the graph, it's reasonable to contemplate that $x(t)$ is bounded.
Even though the graph is hard to interpret, there're still some interesting
discoveries about the relation between $E(t)$ and $\omega^{2}(t)$.

\begin{figure}[H]
\begin{centering}
\protect\caption{Graph of E(t) versus $\omega^{2}(t)$ when $\omega(t)=\sin(t)$}

\par\end{centering}

\centering{}\includegraphics[width=12cm]{Screen Shot 2015-02-09 at 8.27.37 PM.png}
\end{figure}


In this graph, the green curve is the graph of $E(t)$ and the orange
curve is the graph of $\omega^{2}(t)$. As can be seen from the graph,
similar patterns appears on the graph of $E(t)$ periodically with
period equal to about 10s. We also want to point out that whenever
$\omega^{2}(t)$ reaches zero, or local minimum, $E(t)$ reaches a
local minimum and whenever $\omega^{2}(t)$ reaches a local maximum,
$E(t)$ also reaches local minimum. So we put forward our next theorem.
\begin{thm}
If $\omega^{2}(t)$ reaches a local minimum(maximum) at $t=t_{0}$
and $x(t_{0})\neq0,$ $E(t)$ also reaches a local minimum(maximum). \end{thm}
\begin{proof}
As we have shown above, $E'(t)=(\omega^{2}(t))'x^{2}(t)$. Let $g(t)=\omega^{2}(t).$
Let $g(t)=\omega^{2}(t)$. We can write $E'(t)=g'(t)x^{2}(t)$. When
$g(t)$ is at a local maximum and minimum, $g'(t)=0$. Then $E'(t)$=
0. $E''(t)=g''(t)\cdot x^{2}(t)+2g'(t)\cdot x(t)\cdot x'(t)=g''(t)x^{2}(t)$.
When $x^{2}(t)$ is not zero, if $g''(t)>0$ we can get $E''(t)>0$
and if $g''(t)<0$, we have $E''(t)<0$. Based on the second-order
derivative test, we know when if $g(t)$ reaches a local minimum at
$t_{0}$ and $x(t_{0})\neq0$, so does $E(t)$ at $t=t_{0}$ and if
$g(t)$ reaches a local minimum at $t_{0}$ and $x(t_{0})\neq0$,
so does $E(t)$ at $t=t_{0}$. 
\end{proof}
Now we consider the general case of $\omega(t)=\sin(kt)+c$. Here
we present the graphs of $E'(t)$ and $x'(t)$ with $\omega(t)=\sin(kt)+c$
for different $k$'s and $c'$s.

\begin{figure}[H]
\begin{centering}
\protect\caption{$x(t)$ when $\omega(t)=\sin(t)+\pi$}

\par\end{centering}

\centering{}\includegraphics[width=12cm]{x(t)}
\end{figure}


\begin{figure}[H]
\begin{centering}
\protect\caption{$x^{2}(t)$ when $\omega(t)=\sin(t)+\pi$}

\par\end{centering}

\centering{}\includegraphics{x^2(t)}
\end{figure}


\begin{figure}[H]
\begin{centering}
\protect\caption{$E'(t)$ when $\omega(t)=\sin(t)+\pi$}

\par\end{centering}

\centering{}\includegraphics{E'(t)}
\end{figure}


\begin{figure}[H]
\begin{centering}
\protect\caption{$E'(t)$ when $\omega(t)=\sin(t)$}

\par\end{centering}

\centering{}\includegraphics{E'(sin(t))}
\end{figure}


\begin{figure}[H]
\begin{centering}
\protect\caption{$E'(t)$ when $\omega(t)=\sin(t)+1$}

\par\end{centering}

\centering{}\includegraphics{E'(w(t)+1)}
\end{figure}


From then the plots in above, we see that the period of $E$ does
not exactly equal to $2\pi$ for $\omega(t)=\sin t+c$ but it become
increasingly close to $2\pi$ as $c$ increases.

If $\omega(t)=\sin t$, $(\omega^{2})'=\sin2t=2(\sin t\cdot\cos t$),
$E'=\omega\omega'x^{2}=x^{2}(\sin t\cdot\cos t)$. However if $\omega(t)=\sin t+c$,
$(\omega^{2})'=2(\sin t\cdot\cos t+c\cos t)$, $E'=x^{2}(\sin t\cdot\cos t+c\cos t)$.
Since we have $c\cos t$ as a new factor in $E'$, we see that as
$c$ increasing, the factor $c\sin t$ is sufficiently large in $E'$,
which is the major factor in changing the rate of $E$, hence the
shape of $E$ is corresponding to $\int(c\cos t)dt=c\int(\cos t)dt=c\sin t+C$
as $c$ become relative large.

Now we consider some cases of $\omega(x)=\sin t+c$ where $c\neq0$.
\begin{thm*}
(Floquet's Theorem) Suppose A (t) is periodic. Then the principal
matrix solution of the corresponding linear system has the form
\[
\Pi(t,t_{0})=P(t,t_{0})exp((t-t_{0})Q(t_{0})),
\]
 where $P(.,t_{0})$ has the same form as $A(.)$ and $P(t_{0},t_{0})=1$
and $Q(t_{0}+T)=Q(t_{0})$ and $\exp(TQ(t_{0}))=$ $\Pi(t_{0}+T,t_{0})$. \end{thm*}
\begin{fact}
$E(t)$, the energy of the object, is not bounded above for some $\omega=\sin t+c$
where $c\in\mathbb{R}$.

Since $E'=\omega\omega'x^{2}$, then 
\[
\begin{array}{ccc}
E' & = & (\sin t+c)\cos tx^{2}\\
 & = & (\sin t\cos t+c\cos t)x^{2}
\end{array}.
\]



Since $\sin t\cdot\cos t+c\cdot\cos t$ is bounded, then the boundedness
of $E'$ depends on $x^{2}$. Let $y=x'$. We can transfrom the given
equation into $\left(\begin{array}{c}
y'\\
x'
\end{array}\right)+\left(\begin{array}{cc}
0 & \omega^{2}\\
-1 & 0
\end{array}\right)\left(\begin{array}{c}
y\\
x
\end{array}\right)=\left(\begin{array}{c}
0\\
0
\end{array}\right)$. Then we get a two-dimentional periodic lienar system. By Floquet's
theorem, $\left(\begin{array}{cc}
y_{1}(t) & y_{2}(t)\\
x_{1}(t) & x_{2}(t)
\end{array}\right)=P(t,t_{0})\exp((t-t_{0})Q(t_{0}))$. It is also true that any fundamental matrix solution can be written
in the form $V(t)\exp(tR)$ where $V(t)$ is periodic and $R$ is
similar to $Q(t_{0})$. Moreover, note that $Q(t_{0})$ will be complex
even if $A(t)$ is real unless all real eigenvalues of $M(t_{0})$
are positive. However, since A(t) also has the period $2T$ and $\left(\begin{array}{cc}
y_{1}(t+2T) & y_{2}(t+2T)\\
x_{1}(t+2T) & x_{2}(t+2T)
\end{array}\right)=M(t_{0})^{2}$, then we can find a real logarithm for $A^{2}$. 
\begin{casenv}
\item When $c=0$, $\omega(t_{0})=\omega(0)=\sin0+0=0$. Then we have $\left(\begin{array}{cc}
y_{1}(t) & y_{2}(t)\\
x_{1}(t) & x_{2}(t)
\end{array}\right)=V(t)\exp(tR)$. Note $R$ is a constant matrix. Since $\exp(tR)$ is bounded, then
boundedness of the matrix solution is control by $V(t)$. Since $V(t)$'s
entries can be written in the form $\frac{N(t)}{k\cdot\omega(t)}$
where $P'(t)$ is a periodic functions with period $T=2\pi$ and $P'(kT)=0,\forall k\in\mathbb{Z}_{\geq0}$
then $V(t)$ will not blow up $\forall t\geq0$ as $\omega(t)$ and
$P'(t)$ get local min(max) in the same period. 
\item When $c\neq0$, we have $V(t)=\frac{N(t)}{k\cdot\omega(t)+c}$ . Since
the denominator $k\cdot\omega(t)+c=0$, $P'(t)=-c\neq0$. If $k\omega(t)+c>0,\forall t$
then $V(t)$ won't blow up. If $c\in[\min\{k\sin t\},\max\{k\sin t\}\}]$,
$V(t)$ will blow up and we $E'(t_{0})=\infty$ at $t_{0}=kT=2k\pi$.
\end{casenv}
\end{fact}
\begin{example}
$\omega(t)=sin(t)+1$.
\end{example}
Since $V(t)=\frac{N(t)}{ksin(t)+c}$, we get $k\approx0.48$. Hence
$E'$ will blow up if we pick $c\in[1-0.48,1+0.48]$. Since $1$ is
in this interval, then $E$ will definitely blows up. Note $V$ blows
up periodicly at $2k\pi=k\cdot T,k\neq0$. 

\begin{figure}[H]
\begin{centering}
\protect\caption{$E(t)$ and $\omega(t)$ when $\omega(t)=\sin(t)+1$}

\par\end{centering}

\centering{}\includegraphics[width=12cm]{\string"Screen Shot 2015-02-23 at 12.20.56 PM\string".png}
\end{figure}

\begin{thm}
For all piecewise continuous periodic function $a(t)$ with period
$T$ and amplitude $A$, if the solution to the differential equation
$x''(t)+a^{2}(t)x(t)=0$ written in the form of $V(t)\exp(tR)$ such
that $V(t)=\frac{N(t)}{ka(t)+c}$ and$N(t)$ is a periodic function
with period $T$ suffices $\int_{t_{0}}^{t_{0}+kT}a(s)ds=0$ or, $k\neq0$
and $c\geqslant k$, the energy of the system is bounded.\end{thm}
\begin{proof}
Since $E=\frac{(x'(t))^{2}}{2}+\frac{\omega^{2}(t)x^{2}(t)}{2}\geqslant0.$
It is bounded below.

Recall from pervious Fact, we can write $V(t)=\frac{N(t)}{k\cdot a(t)+c}$.
Since $\int_{t_{0}}^{t_{0}+kT}a(s)ds=0$, we have $c=0$, which means
that the denominator $k\cdot a(t)+c=0$ if and only if $N(t)=0$.
Hence $V$ is bounded above in this case. 

If $ka(t)+c$'s amplitude does not intersect with $x$-axis, then
$c\notin[\min\{ka(t)\},\max\{ka(t)\}\}]$. It implies that $V(t)$
will blow up and we have $E'(t_{0})=\infty$ at $t_{0}=kT$. Therefore,
$E$ will not blows up in the two cases stated in the theorem.

\end{proof}
\bibliography{proj1}
\bibliographystyle{amsalpha}   
\end{document}
