

\documentclass[12pt]{article}

\usepackage{amssymb}

\parskip 5pt
\parindent 0pt

\pagestyle{empty}

\begin{document}

{\bf Adiabatic Processes}

\bigskip
Suppose the frequency $\omega$ in a harmonic oscillator drifts slowly:
\begin{equation}
\ddot x+\omega(t)^2x=0
\label{harm}
\end{equation}
where $\omega(t)$ is some slowly varying function.
The energy of the system is the sum of kinetic and  potential energies:
\begin{equation}
E=\frac{\dot x^2}{2}+\frac{\omega^2 x^2}{2}
\label{energy}
\end{equation}

It is constant in time for the harmonic oscillator itself.  The
question is this: What is the effect on $E$ of a slow drift in the
frequency $\omega$ drifts slowly?  Investigate this numerically, using
various functions $\omega(t)$.  One such function should be of the
form $\omega(t)=\omega_0+\epsilon t$, for a small constant
$\epsilon$. Does the effect depend strongly on  the trajectory
$\omega$ follows as it drifts from one value to another?

Does $E$ remain constant, or does it vary systematically with $\omega$?
If for example it increases as $\omega$ increases, this means that
the system absorbs energy as the oscillation is speeded up.
This is what would happen for example if you were to slowly shorten
the length of a pendulum. How does $E$ vary with $\omega$, if it does?
Investigate this in as much detail as you can, and account analytically
for as many observed features as you can.


\end{document}
